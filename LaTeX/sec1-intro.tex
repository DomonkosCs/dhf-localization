
Even in antiquity, localization -- based on various reference points -- played a significant role, especially in the case of pelagic navigation. In the Age of Discovery, the so-called dead reckoning method served as a particularly handy method of position estimation. In this case, localization merely consisted of approximating the traveled distance by only measuring speed, the direction of movement, and elapsed time. Nowadays the de facto standard is the Global Navigation Satellite System (GNSS), however, it behaves poorly (or even not functional) in shielded environments. If for example, an autonomous vehicle performs a parking task in a garage, localization is often performed by different state estimation algorithms.

Such algorithm is the Kalman Filter, which serves as an optimal estimator for linear, Gaussian, dynamical systems. However, in reality, both linearity and Gaussian nature are often violated. To overcome these effects, various nonlinear filters, such as the Extended Kalman filter (EKF) or different particle filter-based algorithms can be applied instead. One prominent alternative is the so-called Daum–Huang filter (DHF), which development was mainly motivated by the shortcomings of particle filters, and was originally designed for satellite localization.

The main goal of this report is to propose an improved
localization algorithm for a mobile robot with an
equipped LiDAR sensor, based on the DHF.
Nowadays, one of the most popular localization algorithm
for such setup is the Adaptive Monte Carlo Localization, which is based on a particle filter.
As Daum--Huang filters are mainly developed to overcome important shortcomings of particle
filters, they may be able to also outperform particle filters in a low-dimensional mobile robot
localization task.

In order to implement the sought localization algorithm, first, certain prerequisites
have to be introduced. In Section 2, the localization task is described in more detail,
followed by the basic structure of the TurtleBot3 mobile robot, along with its available sensors.
As the environment representation and the motion model of the robot are closely related to the equipped sensors, they are also discussed in this section.

Besides the motion model, another existential part of localization is the measurement model.
To suit the DHF algorithm, a newly proposed measurement model by Dantanarayana et al. in \cite{Dantanarayana2013} is utilized and detailed in section 3. Their method provides an implicit
measurement equation which is not suitable for the DHF-based localization in scope, as it
requires an explicit form. As a solution, a small addition is proposed in this report,
which enables the use of an implicit measurement equation.

The extended Kalman Filter (EKF) is directly required for the implemented Daum--Huang filter variant, therefore the Kalman filters are discussed in Subsection 4.1.
As Daum--Huang filters are closely related to particle filters, a brief overview is provided
in Subsection 4.3 through the examination of the bootstrap particle filter. This subsection also contains the description of the AMCL algorithm. Finally, the Daum--Huang filters are discussed in Subsection 4.4, along with the exact flow variant, as it is going to serve as the base for the
proposed localization algorithm, which pseudocode is provided in Subsection 4.5.


In Section 5, the performance of three localization algorithms is going to be tested:
the EKF-based localization  by Dantanarayana et al. \cite{Dantanarayana2016},
the EDH-based localization  algorithm proposed in this report,
and the Adaptive Monte Carlo Localization (AMCL)~\cite{AMCLROS2002}.
Their estimation errors are compared in a simulated environment by the Gazebo simulator and the Robot Operating System.

The report concludes with a summary in Section 6.


